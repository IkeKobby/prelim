%%%%%%%%%%%%%%%%%%%%%%%%%%%%%%%%%%%%%%%%%%%%%%%%%%%%%%
% A Beamer template for University of Wollongong     %
% Based on THU beamer theme                          %
% Author: Qiuyu Lu                                   %
% Date: July 2024                                    %
% LPPL Licensed.                                     %
%%%%%%%%%%%%%%%%%%%%%%%%%%%%%%%%%%%%%%%%%%%%%%%%%%%%%%

\documentclass[serif, aspectratio=169]{beamer}
%\documentclass[serif]{beamer}  % for 4:3 ratio
\usepackage[T1]{fontenc} 
\usepackage{fourier} % see "http://faq.ktug.org/wiki/uploads/MathFonts.pdf" for other options
\usepackage{hyperref}
\usepackage{latexsym,amsmath,xcolor,multicol,booktabs,calligra}
\usepackage{graphicx,pstricks,listings,stackengine}
\usepackage{lipsum}

\author{Isaac Kobby Anni}
\title{BERT: Pre-training of Deep Bidirectional Transformers for Language Understanding}
\subtitle{Paper Presentation}
\institute{
    Computer Science \\
    Bowling Green State University
}
\date{\small \today}
\usepackage{UoWstyle}

% defs
\def\cmd#1{\texttt{\color{brown}\footnotesize $\backslash$#1}}
\def\env#1{\texttt{\color{blue}\footnotesize #1}}
\definecolor{brown}{rgb}{0,0,0.5}
\definecolor{deepred}{RGB}{153,0,0}
\definecolor{deepgreen}{rgb}{0,0.5,0}
\definecolor{halfgray}{gray}{0.55}

\lstset{
    basicstyle=\ttfamily\small,
    keywordstyle=\bfseries\color{deepblue},
    emphstyle=\ttfamily\color{deepred},    % Custom highlighting style
    stringstyle=\color{deepgreen},
    numbers=left,
    numberstyle=\small\color{halfgray},
    rulesepcolor=\color{red!20!green!20!blue!20},
    frame=shadowbox,
}


\begin{document}

\begin{frame}
    \titlepage
    \vspace*{-0.6cm}
    \begin{figure}[htpb]
        \begin{center}
            % \includegraphics[keepaspectratio, scale=0.2]{pic/image.png}
        \end{center}
    \end{figure}
\end{frame}

\begin{frame}    
\tableofcontents[sectionstyle=show,
subsectionstyle=show/shaded/hide,
subsubsectionstyle=show/shaded/hide]
\end{frame}

\section{Background}

\begin{frame}{Prerequisite}
    \begin{itemize}[<+-| alert@+>] % stepwise alerts
        \item Knowledge of recurrent neural networks and how they work.
        \item Attention is all you need paper. 
    \end{itemize}
\end{frame}

\begin{frame}{Introduction}
    \begin{itemize}[<+-| alert@+>] % stepwise alerts
        \item RNNs like LSTM and GRU were established as the state of the art for sequence modeling. 
        \item For problems like language modeling and machine translation. 
    \end{itemize}
\end{frame}

\begin{frame}{Introduction}
    \begin{figure}[h]
            \centering
            \includegraphics[width = 8cm, height = 5cm]{pic/rnn.PNG}
            \caption{Workflow of RNN: \textit{photo credit, Umar Jamil}}
        \end{figure}
\end{frame}

\begin{frame}{Introduction}
    \begin{itemize}[<+-| alert@+>] % stepwise alerts
        \item Problem as sequence length increases; 
        \item[i] Slow computation
        \item[ii] Varnishing and Exploding gradients
        \item[iii] Difficulty in accessing information from longer past time step
    \end{itemize}
\end{frame}


\section{Introducing Transformer}

\subsection{Attention is all you need. }

\begin{frame}{Transformer Architecture}
    \begin{figure}[h]
            \centering
            \includegraphics[width = 10cm, height = 6cm]{pic/trans.PNG}
            \caption{Transformer Model: \textit{photo credit, Attention paper}}
        \end{figure}
\end{frame}

\begin{frame}{The Encoder}
    \begin{figure}[h]
            \centering
            \includegraphics[width = 10cm, height = 6cm]{pic/encoder.PNG}
            \caption{Transformer Model: \textit{photo credit, Attention paper}}
        \end{figure}
\end{frame}

\begin{frame}{Input Embedding}
    \begin{figure}[h]
            \centering
            \includegraphics[width = 13cm, height = 1cm]{pic/embed-1.PNG}
            \caption{Document Representation: \textit{photo credit, Umar Jamil}}
    \end{figure}
\end{frame}
\begin{frame}{Input Embedding}
    \begin{figure}[h]
            \centering
            \includegraphics[width = 13cm, height = 2cm]{pic/embed-2.PNG}
            \caption{Document Representation}
    \end{figure}
\end{frame}
\begin{frame}{Input Embedding}
    \begin{figure}[h]
            \centering
            \includegraphics[width = 13cm, height = 4cm]{pic/embed-3.PNG}
            \caption{Document Representation}
    \end{figure}
\end{frame}

\begin{frame}{Positional Embedding}
    \begin{figure}[h]
            \centering
            \includegraphics[width = 12cm, height = 7cm]{pic/pos-end.png}
            \caption{Document Representation}
    \end{figure}
\end{frame}

\begin{frame}{Positional Embedding}
    \begin{itemize}[<+-| alert@+>] % stepwise alerts
        \item Addressing the sequential encoding like in RNNs; 
        \item[i] Position of each word in the document. 
        \item[ii] Treating words that are close as "close" and distant as "distant"
        \item[iii] Position encoding portray patterns learned by the model. 
    \end{itemize}
\end{frame}

\begin{frame}{Positional Embedding}
    \begin{figure}[h]
            \centering
            \includegraphics[width = 13cm, height = 2cm]{pic/embed-2.PNG}
            \caption{Document Representation}
    \end{figure}
\end{frame}
\begin{frame}{Positional Embedding}
    \begin{figure}[h]
            \centering
            \includegraphics[width = 13cm, height = 5cm]{pic/pos_e.png}
            \caption{Document Representation}
    \end{figure}
\end{frame}

\begin{frame}{Positional Embedding Calculation}
    \begin{figure}[h]
            \centering
            \includegraphics[width = 13cm, height = 6cm]{pic/pos_encod.png}
            \caption{Document Representation}
    \end{figure}
\end{frame}

\begin{frame}{Attention Mechanism}
    \begin{figure}[h]
            \centering
            \includegraphics[width = 12cm, height = 7cm]{pic/attn.png}
            \caption{Multi-Head Attention}
    \end{figure}
\end{frame}
\begin{frame}{Attention Mechanism}
    \begin{figure}[h]
            \centering
            \includegraphics[width = 13cm, height = 5cm]{pic/self-attn.png}
            \caption{Self Attention computation}
    \end{figure}
\end{frame}
\begin{frame}{Attention Mechanism}
    \begin{figure}[h]
            \centering
            \includegraphics[width = 13cm, height = 5cm]{pic/self-attn-2.png}
            \caption{Self Attention computation}
    \end{figure}
\end{frame}

\begin{frame}{Properties of Self Atention}
    \begin{itemize}[<+-| alert@+>]
        \item No requires no parameter(s)
        \item Values along diagonals are higher. 
        \item Can switch off word interactions. etc
    \end{itemize}
\end{frame}

\begin{frame}
    \frametitle{Self-Attention Computation}
    
    \begin{minipage}{0.48\linewidth}
        \centering
        \includegraphics[width=5cm, height=4cm]{pic/softmax.png}
        %\captionof{fig}{Softmax function}
    \end{minipage}%
    \hfill
    \begin{minipage}{0.48\linewidth}
        \centering
        \includegraphics[width=5cm, height=5cm]{pic/attn-calc.png}
        %\captionof{ }{Attention computation}
    \end{minipage}
    
\end{frame}

\begin{frame}{Attention Mechanism}
    \begin{figure}[h]
            \centering
            \includegraphics[width = 13cm, height = 5cm]{pic/MHA.png}
            \caption{Multi-Head Attention}
    \end{figure}
\end{frame}

\begin{frame}{Multi-Head Attention Mechanism - DECODER}
    \begin{figure}[h]
            \centering
            \includegraphics[width = 14cm, height = 5cm]{pic/decoder.png}
            \caption{Masked Multi-head Attention computation}
    \end{figure}
\end{frame}

\section{BERT}

\begin{frame}{Motivation}
    \begin{itemize}[<+-| alert@+>]
        \item Existing models like OpenAI GPT were unidirectional.
        \item Understanding context in both directions was a challenge.
    \end{itemize}
\end{frame}

\begin{frame}{Goal}
    \begin{itemize}
        \item To build a model that fully leverages context from both directions for improved language understanding.
    \end{itemize}
\end{frame}

\begin{frame}{Model Architecture}
    \begin{itemize}[<+-| alert@+>]
        \item Architecture is made of layers of \textbf{ENCODER}'s of the transformer model. 
        \item \textbf{BERT-base}: 
            \begin{itemize}[<+-| alert@+>]
                \item[i] 12 encoder layers
                \item[ii] 12 attention heads
                \item[iii] 3072 hidden size of feed-forward layer
                \item[iv] 768 embedding size
            \end{itemize}
    \end{itemize}
\end{frame}

\begin{frame}{Model Architecture}
    \begin{itemize}[<+-| alert@+>]
        \item \textbf{BERT-large: }
        \item[i] 24 encoder layers
        \item[ii] 16 attention heads
        \item[iii] 4096 hidden size of feed-forward layer
        \item[iv] 1024 embedding size
    \end{itemize}
\end{frame}

\begin{frame}{BERT Training}
Two ways of training;
    \begin{itemize} [<+-| alert@+>]
        \item[1] Pre-training
        \item[2] Fine-tuning
    \end{itemize}
\end{frame}


\begin{frame}{Experiment \& Results}
    \begin{figure}[h]
            \centering
            \includegraphics[width = 13cm, height = 5cm]{pic/experiment.png}
            \caption{Experiment and results}
    \end{figure}
\end{frame}

\begin{frame}{Limitations}
    \begin{itemize} [<+-| alert@+>]
        \item Computationally expensive.
        \item Requires large datasets.
        \item Limited understanding of rare or out-of-distribution words
    \end{itemize}
\end{frame}

\begin{frame}{Conclusions}
    \begin{itemize} [<+-| alert@+>]
        \item BERT revolutionized NLP with bidirectional transformers.
        \item Future improvements focus on efficiency (e.g., DistilBERT) and domain-specific adaptations (e.g., BioBERT).
    \end{itemize}
\end{frame}

% \section{References}

% \begin{frame}[allowframebreaks]
%     \bibliography{ref}
%     \bibliographystyle{ieeetr}
%     \nocite{*} % used here because no citation happens in slides
%     % if there are too many try use:
%     % \tiny\bibliographystyle{alpha}
% \end{frame}


\begin{frame}
    \begin{center}
        {\Huge\calligra Thank You}
    \end{center}
\end{frame}

\end{document}